\documentclass[12pt]{report}

\usepackage[a4paper, margin=2cm]{geometry}
\usepackage[french]{babel}
\usepackage[utf8]{inputenc}
\usepackage[T1]{fontenc}
\usepackage{graphicx}
\usepackage{booktabs}
\usepackage{hyperref}
\usepackage{fancyhdr}
\usepackage{enumitem}
\usepackage{titlesec}
\usepackage{import}
\usepackage{alltt}
\usepackage{longtable}
\usepackage{fancyvrb}
\usepackage{listings}
\usepackage{xcolor}

\colorlet{punct}{red!60!black}
\definecolor{background}{HTML}{EEEEEE}
\definecolor{delim}{RGB}{20,105,176}
\colorlet{numb}{magenta!60!black}

\lstdefinelanguage{json}{
    basicstyle=\normalfont\ttfamily,
    numbers=left,
    numberstyle=\scriptsize,
    stepnumber=1,
    numbersep=8pt,
    showstringspaces=false,
    breaklines=true,
    frame=lines,
    backgroundcolor=\color{background},
    literate=
     *{0}{{{\color{numb}0}}}{1}
      {1}{{{\color{numb}1}}}{1}
      {2}{{{\color{numb}2}}}{1}
      {3}{{{\color{numb}3}}}{1}
      {4}{{{\color{numb}4}}}{1}
      {5}{{{\color{numb}5}}}{1}
      {6}{{{\color{numb}6}}}{1}
      {7}{{{\color{numb}7}}}{1}
      {8}{{{\color{numb}8}}}{1}
      {9}{{{\color{numb}9}}}{1}
      {:}{{{\color{punct}{:}}}}{1}
      {,}{{{\color{punct}{,}}}}{1}
      {\{}{{{\color{delim}{\{}}}}{1}
      {\}}{{{\color{delim}{\}}}}}{1}
      {[}{{{\color{delim}{[}}}}{1}
      {]}{{{\color{delim}{]}}}}{1},
}

\setlength{\headheight}{30pt}
\setlength{\parskip}{1em}  
\setlength{\parindent}{0pt} 
\titleformat{\chapter}[hang]{\Huge\bfseries}{\thechapter.}{2ex}{}
\titleformat{\section}[hang]{\large\bfseries}{\thesection}{1em}{}

\title{Rapport de projet - GEPO \\ \large Laboratoire 01}
\author{Hazeraj Herdison}
\date{\today}

\pagestyle{fancy}
\fancyhf{}
\rhead{GEPO}
\lhead{Laboratoire 01}
\rfoot{\thepage}

\begin{document}

\maketitle
\tableofcontents
\newpage

\import{./parts/}{introduction-gepo.tex}
\import{./parts/}{mise_en_place-gepo.tex}
\import{./parts/}{claude-gepo.tex}
\import{./parts/}{mcp-gepo.tex}
\import{./parts/}{express-gepo.tex}

\section{Github}
Cette partie contient toutes les informations relatives à mon dépôt Github. 

\import{./scripts/}{git_report.tex}

\subsection*{Exemple de tableau}
\begin{center}
\begin{tabular}{lll}
\toprule
\textbf{Tâche} & \textbf{Responsable} & \textbf{Statut} \\
\midrule
Analyse        & Mobutu     & Terminé \\
Développement  & Équipe Dev & En cours \\
Tests          & QA Team    & À venir \\
\bottomrule
\end{tabular}
\end{center}

\section{Illustration}
Voici un exemple d'insertion d'image :

\begin{figure}[h!]
\centering
\includegraphics[width=0.6\textwidth]{example-image}
\caption{Exemple d'image dans un rapport}
\end{figure}

\section{Conclusion}
Ce modèle montre les bases d’un rapport en LaTeX. Il est possible d’ajouter des annexes, des citations, des listes, des codes sources, et bien plus.

\newpage
\section*{Bibliographie}
\begin{thebibliography}{9}
\bibitem{lamport}
Leslie Lamport,
\textit{LaTeX: A Document Preparation System},
Addison-Wesley, 2nd edition, 1994.

\bibitem{latex3}
LaTeX Project,
\textit{Official LaTeX Project Website},
\url{https://www.latex-project.org}
\end{thebibliography}

\end{document}
